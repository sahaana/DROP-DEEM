
\begin{abstract}
Dimensionality reduction is crucial to a number of data analytics workflows, such as anomaly detection, classification, and predictive analysis, and has been extensively studied in a variety of domains, including statistics, signal processing and data mining. Recently, there has been an increase in the amount of data generated from automated sensors, devices and processes, collectively, the Internet of Things (IoT). It is becoming increasingly important for data analytics techniques to scale to these immense volumes of IoT, often time-series, data to extract value from them in a timely fashion. 

Analysis of various dimensionality reduction techniques reveals that given an accuracy bound on pairwise distance preservation, Principal Component Analysis (PCA) outperforms most other techniques with respect to accuracy, but is often orders of magnitude slower than other techniques. However, by exploiting the structure present in automatically generated IoT and time series data, we develop a new, sampling-based technique for automatically and efficiently finding the optimal low-dimensional basis for a given time-series and/or IoT dataset---DROP (Dimensionality Reduction OPtimizer). DROP iteratively subsamples the dataset and searches for an $\epsilon$-distance-preserving transformation.
For time-series and IoT data with limited variance, this allows DROP to uncover the optimal basis in running time that is \textit{independent} of the actual dataset size and \textit{without} requiring the user to specify the intrinsic dimensionality of the dataset a priori; instead, DROP discovers it. This provides order of magnitude speedups over na{\"i}ve PCA and blah blah.  


 \end{abstract}
