\section{Background and Problem}
\label{sec:background}

In this section, we  provide background on dimensionality reduction (DR), and define our problem of workload-aware dimensionality reduction (DR). 

\subsection{Dimensionality Reduction}
\label{sec:defs}

DR refers to finding a low-dimensional representation of a dataset that preserves properties of interest, such as data point similarity~\cite{dr-survey1,dr-survey2}. Formally, consider a data matrix $X \in \mathbb{R}^{\mvar \times \dvar}$, where each row $i$ corresponds to data point $x_i \in \mathbb{R}^\dvar$, with $\mvar > \dvar$.  
DR computes a transformation function ($T: \mathbb{R}^\dvar \rightarrow \mathbb{R}^k$) that maps each $x_i$ to a new basis as $\tilde{x}_i \in \mathbb{R}^k$ where $k \leq \dvar$, resulting in a new data matrix $T(X) = \tilde{X} \in \mathbb{R}^{\mvar \times k}$.

\subsubsection*{Principal Component Analysis (PCA)}
\label{sec:pca}
PCA is a linear DR technique that identifies a new orthogonal basis for a dataset that captures its directions of highest variance.
Of all linear transformations, this basis minimizes reconstruction error in a mean square sense. 
Classically implemented PCA uses a Singular Value Decomposition (SVD) routine~\cite{trefethen}.

\begin{comment} 
\begin{algorithm}
\begin{algorithmic}[1]
\Statex \textbf{Inputs:}  
\Statex $X \in \mathbb{R}^{m_1 \times d}$: training data matrix 
\Statex $Y \in \mathbb{R}^{m_2 \times d}$: data matrix to transform 
\Statex $k \in \mathbb{Z}_+$: desired dimensionality of transformed data
\Statex \textsc{SVD-T}: any truncated SVD algorithm  
\Statex
\Statex \hrule
\Function{fit}{$X$}:
	\State $\bar{X} = \text{columnMeans}(X)$
	\State $C_X = X - \bar{X}$
		\Comment{$C_X \in \mathbb{R}^{m_1 \times d}$}
	\State \textbf{Store: } $\bar{X}, C_A$
\EndFunction

\Function{transform}{$Y, k, $ \textsc{SVD-T}}:
	\State $U, \Sigma, V^T$ = \textsc{SVD-T}$(C_X, k)$
		\Comment{$V \in \mathbb{R}^{d \times k}$}
	\State $C_Y = Y - \bar{X}$
		\Comment{$C_Y \in \mathbb{R}^{m_2 \times d}$} 
	\State \textbf{Store: } $T = V$ 
			\Comment{Cache for repeated use} \\
	\Return $C_YT$
		%\Comment{$C_BV \in \mathbb{R}^{M_2 \times k}$} 
\EndFunction

\end{algorithmic}
\caption{PCA via truncated SVD}
\label{alg:PCA-inc}
\end{algorithm}

%discuss advanced techniques
As described in Section~\ref{related}, several theoretical advances provide accelerated means of of efficiently computing PCA over large-scale data beyond the na\"ive SVD-based approach.
These methods operate on samples of input data, and---in theory---confer substantial runtime benefits when in fact a low-dimensional basis exists (i.e., the spectrum of eigenvalues has a large drop).
However, there are two main challenges in utilizing these approaches.
First, it is unclear when to stop sampling data points when using stochastic or mini-batch methods, including state-of-the-art momentum techniques that achieve accelerated convergence rates~\cite{CDS}.
This is because convergence of these techniques (e.g., the magnitude of the gradient in stochastic gradient methods) does not correspond directly to preservation of metrics of interest.
Second, these techniques typically rely the target reduced dimension ($k$) to be specified a priori, but the suitable $k$ for the task at hand is rarely known a priori for a given dataset. The choice of $k$ can dramatically affect runtimes and convergence rates, making the target dimensionality an important, yet difficult to obtain parameter. 

Thus, even with advanced techniques, it is unclear \emph{how much computation is required} to obtain acceptable low dimensional representations, and \emph{how low a dimension is considered acceptable} for specific application constraints. 
We describe how DROP overcomes these challenges in [forward ref sampling], and show how answering these questions enables improvements over previous techniques when evaluated in an end-to-end context.  
\end{comment}

\subsection{DR \red{for Repeated-Query Workloads}}

In workloads such as similarity search, clustering, \red{or classification}, ML models are periodically trained over historical data, and are \emph{repeatedly queried} as incoming data arrives or new query needs arise (see Fig~\ref{fig:pipeline}). 
Indexes built over this data can improve the efficiency of this repeated query workload in exchange for a preprocessing overhead.
DR with a multidimensional index structure is a classic way of achieving this, and is the basis for popular similarity search procedures and extensions in the data mining and machine learning communities~\cite{keogh-indexing,local-dr,charu-ss,dynamic-ss,dm-book,humming-index,decade,search}; a metric-preserving transformation reduces input dimensionality, and an index is built in this new space for subsequent queries.


\subsubsection*{\red{DR in Similarity Search}}
\red{Similarity search is a common repeated-query workload performed over a variety of data types including images, documents and time series~\cite{keogh-study,lsh}, which we use as a running case study given its popularity and the large amount of research in the space.
The Tightness of Lower Bounds ($TLB$) is typically the metric preserved by DR in this setting~\cite{keogh-study}, as it measures how well a \emph{contractive} DR transformation (i.e. distances in the transformed space are less than or equal to those in the original) preserves pairwise Euclidean distances. It can be used to identify the quality of a low dimensional transformation without performing the downstream similarity search task:}
\begin{equation}
\label{eq:tlb}
TLB = \frac{2}{\mvar(\mvar-1)}\sum_{i<j}\frac{\| \tilde{x}_i -  \tilde{x}_j \|_2 }{\| x_i -  x_j\|_2 }.
\end{equation}


\subsection{Problem: Workload-Aware DR}
\label{subsec:wadr}

In workload-aware dimensionality reduction, we perform DR to minimize overall workload runtime.
DR is a fixed cost (i.e., in index construction for similarity search), while each query (i.e., nearest neighbor query) over the dataset incurs a marginal cost that is dependent on DR quality: lower-dimensional data points result in faster queries. 

As input, consider a set of data points, a desired level of metric preservation ($B$; e.g., $TLB \geq .99$) and, optional downstream runtime as a function of dimensionality ($\mathcal{C}_\mvar(\dvar)$ for an $\mvar\times \dvar$ data matrix).  
We use this information to efficiently return a DR function that satisfies the metric constraint with a configurable degree of confidence.
More formally, denoting DR runtime as $R$, we define the optimization problem:
\begin{problem}
\label{def:opt}
  Given $X \in \mathbb{R}^{\mvar \times \dvar}$, $TLB$ constraint $B \in 
  (0, 1]$, confidence $c$, and workload runtime function $\mathcal{C}_\mvar:\mathbb{Z}_{+} \rightarrow \mathbb{R}_{+}$, find $k$ and transformation
  matrix $T_k \in \mathbb{R}^{\dvar \times k}$ that minimizes $R + \mathcal{C}_\mvar(k)$
  such that $TLB(XT_k) \geq B$ with confidence $c$.
\end{problem}

We assume the downstream runtime model $\mathcal{C}_\mvar(\dvar)$ is monotonically increasing in $\dvar$ as the premise of DR for efficient analytics relies on downstream tasks running faster on lower dimensional data.
Absent this, we default to execution until convergence (i.e, until $k$ plateaus) as described in Section~\ref{sec:sampling}, and demonstrate the cost of doing so in Section~\ref{sec:experiments}.

The more time spent on DR ($R$), the smaller the transformation ($k$), thus the lower the workload runtime.
To minimize $R + \mathcal{C}_\mvar(k)$, we must determine how much time to spend on DR to minimize end-to-end runtime.
